
%----------------------------------------------------------------------------------------
%	PACKAGES AND OTHER DOCUMENT CONFIGURATIONS
%----------------------------------------------------------------------------------------

\documentclass[12pt]{article}
\usepackage{appendix}
\usepackage[english]{babel}
\usepackage{minted}
\usepackage{amsmath}
\usepackage{graphicx}
\usepackage{float}
\usepackage[colorlinks = true,
linkcolor = black,
urlcolor  = blue,
citecolor = blue,
anchorcolor = blue]{hyperref}
\usepackage{blindtext}
\usepackage[table,xcdraw]{xcolor}
\usepackage{listings}
\usepackage{color}
\definecolor{light-gray}{gray}{0.95}
\lstset{
	numbers=left,
	breaklines=true,
	backgroundcolor=\color{light-gray},
	tabsize=2,
	basicstyle=\ttfamily,
}
\newminted{dart}{
	fontsize=\footnotesize, 
	escapeinside=||,
	mathescape=true,
	numbersep=5pt,
	linenos=true,
	gobble=2,
	framesep=3mm} 
\usepackage{hyperref}
\hypersetup{linktoc = all}
\renewcommand{\arraystretch}{1.5}	% Increase row space between tables



\begin{document}

	
	\begin{titlepage}
		
		\newcommand{\HRule}{\rule{\linewidth}{0.5mm}} 
		
		\center % Center everything on the page
		
		%----------------------------------------------------------------------------------------
		%	HEADING SECTIONS
		%----------------------------------------------------------------------------------------
		\textsc{\Large MSc Project: Mobile Hairdresser Application}\\[5 cm] 
		
		
		
		\includegraphics[scale=0.05]{images/logo.png}\\[1 cm]
		
		
		
		
		
		\textsc{\LARGE Joshua Robertson} \\[6 cm]
		
		
		
		
		
		%----------------------------------------------------------------------------------------
		%	TITLE SECTION
		%----------------------------------------------------------------------------------------
		
		
		
		%----------------------------------------------------------------------------------------
		%	AUTHOR SECTION
		%----------------------------------------------------------------------------------------
		
		
		\begin{center}
			
			\textsc\emph{{“A dissertation submitted to the University of Bristol in accordance with the requirements of the degree of Master of Science by advanced study in Computer Science in the Faculty of Engineering."}} \\[1.2 cm]
			
			School of Computer Science, Electrical and Electronic Engineering, and Engineering Maths (SCEEM) \\[1 cm]
			
			
			
		\end{center}
		
		
		
		
		\vfill % Fill the rest of the page with whitespace
		
	\end{titlepage}


	\section*{Abstract}
	(300 words)
	
	\pagebreak

	\section*{Acknowledgements}
	\pagebreak
	
	\section*{Author's Declaration}

	I declare that the work in this dissertation was carried out in accordance with the 
	requirements of the University’s Regulations and Code of Practice for Taught Programmes 
	and that it has not been submitted for any other academic award.  Except where indicated 
	by specific reference in the text, this work is my own work. Work done in collaboration with, 
	or with the assistance of others, is indicated as such. I have identified all material in this 
	dissertation which is not my own work through appropriate referencing and 
	acknowledgement. Where I have quoted or otherwise incorporated material which is the 
	work of others, I have included the source in the references.  Any views expressed in the 
	dissertation, other than referenced material, are those of the author. 
	\\
	\\
	SIGNED: Joshua Robertson DATE: 14.09.2021 
	
	\pagebreak
	
	\tableofcontents
	\pagebreak
	
	\listoffigures
	\pagebreak
	
	\section{Introduction}


	
	
	The COVID-19 pandemic has had a ubiquitous impact on our lives, from work, to travel, to how we socialise. Some of these changes are temporary, such as mask wearing, but others, such as the adoption of more digital services and flexible homeworking seem likely to be permanent (ADD CITE). 
	Subsequently, the restriction of movement has caused consumers to migrate to online services at an unprecedented rate. With the global home services market expected to grow by almost 20\% per year until 2026 \cite{owen-ray} there exists a wealth of opportunity for companies to capitalise through digitalising previously physical services. One of these services is hairdressing, which saw a sizeable uptake in demand throughout the pandemic. 
	\\
	
	There already exists a range of applications suited towards providing home haircuts. For example, Shortcut \cite{shortcut}, TRIM-IT\cite{trim-it} and TrimCheck\cite{trimcheck} all provide bookable home haircuts. Despite this, none of the aforementioned applications follow the "uber" model, of allowing for immediate booking and delivery of home haircuts. 
	\\
	
	This application will therefore aim to facilitate this, along with exploring any other market gaps through early research and will aim to achieve this through the following objectives:
	\\
	
	\noindent
	$\bullet$ Conduct research to elucidate any market gaps
	\\
	$\bullet$ Through a user centric methodology design (UCD) and prototype the user interface of the application
	\\
	$\bullet$ Create a minimum viable product (MVP) using new product development
	\\
	
	The research aspect of the project will aim to analyse the strength and weaknesses of existing applications within the field, therefore exposing any market gaps that may be present. 
	
	For the implementation of the application, a heavy focus on the end user was taken through using a UCD methodology, along with NPD, which refers to the  entirety of processes leading to bringing a product to market and encompasses several steps as seen in figure \ref{fig:npd} below and discussed throughout the proceeding sections.
	\newline
	
	\begin{figure}[H]
		\centering
		\includegraphics[scale=0.15]{images/npd.png}
		\caption{The 7 Steps of New Product Development}
		\label{fig:npd} \cite{shopify}
	\end{figure}
	
	
	\subsection{Ideation and Concept}
	The first stage of NPD starts with conceptualisation of a product idea. For the application the initial concept came from an external partner, who proposed a mobile hairdresser application to capitalise on the increased need for home delivery services. This was further defined during several meetings carried out in the initial stages of the project. Through these meetings and discussion with other students, family and friends the following core functionality was created:
	\\
	The application should allow the user to login, select from a range of barbers and book a home haircut.
	
	\subsection{Project Management}

	The project management for the application focused on two key aspects; a clear vision and scope, including a detailed project plan; and an execution phase, which utilized an agile methodology.
	
	\subsubsection{Vision and Scope}
	
	
	\subsubsection{Agile Methodology}
	To carry out the project, an agile methodology was utilised, which allowed for short, iterative cycles of production, providing value in the form of quick creation and constant revision. Taking on an agile methodology also served the project well in the sense that there was a strong focus on prioritising providing value to the end user over comprehensive documentation and lengthy processes. 
	
	Although this approach is more commonly relevant to a team of developers, approaching the project management in this way allowed for a stringent and well defined timeline to be used, aiding in project delivery and outcome. This involved several key stages.
	Firstly, individual 'epics' were defined, which included:
	\\
	$\bullet$ Define the Scope and Market Research
	\\
	$\bullet$ Design and Architect the Application
	\\
	$\bullet$ Setup and Create The Backend
	\\
	$\bullet$ Write the Dissertation
	\\
	
	These were then used to create 'stories' which were further split into individual tasks placed into a timeline and carried out in . For this, the project management tool monday.com \cite{monday.com} was used, which can be seen in figure \ref{fig:monday.com}. Using Monday.com allowed for a timeline to be easily created, along with updating the status of each story when relevant to follow the completion of the project. In this way, a change management approach (ADD CITE) could be carried out, in which the project could easily be updated and reflected in the tasks. This was especially important as both agile methodology and the UCD rely on constant feedback from the end user which informs the project.
	
	\begin{figure}[H]
		\centering
		\includegraphics[scale=0.25]{images/monday.png}
		\caption{Screenshot of Monday.com Displaying the Project Stories}
		\label{fig:monday.com}
		\cite{monday.com}
	\end{figure}

	Finally, using the created timeline, a gantt chart was implemented, which gave an overarching view of the project, with tasks performed represented along the vertical axis and the timeline represented along the horizontal axis. This can be found in the appendix (ADD CITE).
	
	\subsection{Research and Market Analysis}
	In order to gauge whether there is a market for the proposed analysis, a survey was carried out in which users were asked about whether they could see themselves using the application features, among other things. The full survey can be found within the appendix (ADD CITE)
	
	\subsubsection{Existing Applications}
	As previously discusses there exists a variety of similar applications, for which the most prominent will be discussed below, along with the salient and limiting features of each.
	\\
	
	\noindent
	\underline{Shortcut}
	\\
	Shortcut is a US based application..
	One of the defining features of the shortcut app is the availability, with the app providing the ability to request a haircut from 8am to as late as midnight capatilising on a previously unventured late night hair cut market.
	
	The application is limited on it's features, with it only having the option to request a Hair Cut only or a Hair Cut and Beard Trim. 
	Another limiting feature of the application is the price. For a single haircut the cost starts at 75\$ (around £54), which is most likely a reflection of high start up costs and is a problem seen in other similar applications, such as uber and lyft that can only be mitigated through losses (\cite{tech-crunch}).
	\\
	
	\noindent
	\underline{TRIM-IT}
	https://www.bbc.co.uk/news/stories-47711610
	\\
	TRIM-IT is a UK based mobile hairdressor application. Their business model is franchise based similar to that seen by Mcdonald's, whereby barbers would invest through a monthly fee and be provided with the tools necessary, such as a mobile barber unit. 
	
	\noindent
	\underline{TrimCheck}
	
	
	\subsubsection{Novelty of the Proposed Application}
	As discussed in the previous sections, there exists a range of applications that are suited towards providing a mobile barber service.
	
	(shortcut)The ability to offer a variety of services not limited to just a haircut or beardtrim.
	
	\subsection{Deciding on a Platform}
	\subsubsection{Mobile vs Desktop}
	
	\subsubsection{Frontend: Android vs iOS}
	An important consideration when creating a mobile application is deciding on which platform to choose. The two largest mobile providers currently are android and apple (iOS). Historically, iOS has dominated the market share, with a 42.02\% market share in January 2011 compared to Androids 12.42\% (figure \ref{fig:ios-android}). Despite this, in recent years android OS has become more popular, even holding a greater share several times over the last few years and currently trails by only around 2\%.
	
	\begin{figure}[H]
		\centering
		\includegraphics[scale=0.4]{images/ios-android.png}
		\caption{iOS vs Android Market Share Over The Last 10 Years}
		\label{fig:ios-android}
		\cite{stat-counter-21}
	\end{figure}
	
	With this change has brought with it a push towards frameworks that allow for development across multiple platforms, such as React Native (\cite{https://reactnative.dev/}) and Flutter (\cite{flutter.dev}). For this reason, it was decided that a cross platform would be used, which is further discussed below.
	
	\subsection{Frontend: Programming Language}
	\subsubsection{Programming Language}
	When deciding on the programming software, several metrics were taken into consideration, including cross-platform functionality, speed, speed of development and performance. For this reason, Dart and the corresponding Flutter software development kit (SDK) were chosen for the primary software. Flutter is a cross-platform development kit, meaning that it will natively run on both iOS and android applications created by Google \cite{flutter}. Dart is compiled ahead-of-time into native ARM code giving better performance compared to other similar development kits, such as React Native and the user interface  is implemented within a fast, low-level C++ library giving great speed to the application. Dart has also seen a large increase in usage within recent years, jumping up 532\% from 2018 to 2019 \cite{Github, 2018} meaning that there is now an extensible list of third-party plugins available and a large community.
	
	\subsection{Backend: SQL vs noSQL database}
	For the database, it was decided to use Google Firestore (\cite{firestore}), a noSQL database that relies on  nested 'documents' within 'collections'. This was chosen for several reasons. Firstly, as the chosen language 'Dart' is run by Google, using firestore allows for greater integration and congruence with the platform and APIs. Firestore also allows for rapid scalability, along with using Googles excellent cloud platform. The cloud firestore also integrates well with Firebases Authentication service, which is used throughout and discussed extensively in \autoref{chap:implementation}
	
	Another important feature of noSQL databases is the ability to easily modify the internal data in response to changing business requirements, in an interactive way that allows fordo you use relationship data in firebase stackoverflow modification throughout the application lifestyle and therefore easy scaling.
	
	

	
	
	\subsubsection{The Target User}
	

	
	
	\subsection{User Personas}
	\label{user-personas}
	The creation of user personas representing fictitious, archetypal users is an essential part of application development \cite{Grudin and Pruitt, 2002} and allows a deep understanding of the target user to be sought and implemented within the features and design of the application \cite{Long, 2009}. There are, however, some shortcomings to qualitative persona generation, such as validity concerns and user bias \cite{Chapman and Milham, 2007} and although they are addressed by other methods, such as data-driven personas \cite{Mcginn and Kotamraju, 2008}, these require a broad user base and therefore we have decided to stick with qualitative methods, which allow for enough brevity and depth for the scope of the project.
	
	Here 3 user personas were created, which are discussed in detail below.
	\begin{itemize}
		\item Persona 1: 
		
		Name: Sarah Johnson
		
		Profile: 
		\begin{figure}[H]
			
			\includegraphics[scale=0.2]{images/persona_1.png}
			\caption{Persona 1}
			\label{fig:persona_1}
		\end{figure}
	
		(TODO: personas share same name)
		
		\item Persona 2:
		
		Name:  Claire Sheppard
		
		
		Profile: 
		\begin{figure}[H]
			
			\includegraphics[scale=0.2]{images/persona_1.png}
			\caption{Persona 2}
			\label{fig:persona_2}
		\end{figure}
		
		\item Persona 3:
		
		Name: Emma Bradford
		
		Profile: 
		\begin{figure}[H]
			
			\includegraphics[scale=0.2]{images/persona_1.png}
			\caption{Persona 3}
			\label{fig:persona_3}
		\end{figure}
		
		\item Persona 4:
		
		Name:  Claire Sheppard
		
		
		Profile: 
		\begin{figure}[H]
			
			\includegraphics[scale=0.2]{images/persona_4.png}
			\caption{Persona 4}
			\label{fig:persona_4}
		\end{figure}
		
	\end{itemize}
	
		

	\section{System Design}
	This section discusses the process of the ideation and creation of the system design, including the initial scope, requirements, application architecture and state management.
	\begin{figure}[H]
		\centering
 		\includegraphics[scale=0.2]{images/placeholder.jpg}
		\caption{Application Minimal Design Prototype}
		\label{fig:minimal-prototype}
	\end{figure}

	(TODO: Add prototype)

	\subsection{Application Scope}
	The scope of the project was to create a fully working and functional barber application with several features, which are discussed in the User Needs below. To further aid in scoping the project, epics were created, which were further split into stories that could be carried out. Although it could be argued that this type of agile methodology is more relevant when working within a team of developers, it helped to determine a stringent workflow and timeline and aided in project delivery.
	The scope was further defined when wire-framing in Adobe XD, which allowed for the first tangible design to be made.
	
	\subsection{User Needs}
	$\bullet$ Allow the application to run on a mobile device.
	\\
	$\bullet$ Allow the user to book a beauty treatment to receive at their home address.
	\\
	$\bullet$ Allow a barber to set up an account and specify their product details.

	\subsection{Requirements}
	\subsubsection{Functional Requirements}
	
	\begin{itemize}
		\item Customer-side Application
		\begin{itemize}
			\item The application shall allow a customer to create an account and login
			\item The application shall provide the user with basic account management capabilities
			\item The application shall allow for the user to pick from a range of relevant products and add them to their cart
			\item The application shall allow the user full management of their shopping cart
			\item The application shall provide only geographically relevant barbers and products to the user
			\item The application shall allow the user to view their past orders
			
		\end{itemize}
	\end{itemize}

	\begin{itemize}
		\item Barber-side Application
		\begin{itemize}
			\item The application shall allow a parent barber to create an account and login
			\item The application shall allow for integrated back-end management of it's barbers and products
			\item The application shall allow for the parent barber to view its orders
		\end{itemize}
	\end{itemize}
	
	\subsubsection{Non-Functional Requirements}
	
	\begin{itemize}
		\item Performance
		\begin{itemize}
			\item The application shall take no longer than 3 seconds to load the users home screen
		\end{itemize}
	\end{itemize}

	\begin{itemize}
	\item Data
	\begin{itemize}
		\item The application shall cache data where possible
		\item The application shall minimise calls to the database and make them only when relevant
	\end{itemize}
	\end{itemize}

	\begin{itemize}
		\item Use-ability
		\begin{itemize}
			\item The application shall follow nielsen's usability heuristics and be easily usable for the user without any guidance or help
		\end{itemize}
	\end{itemize}

	\begin{itemize}
		\item Security
		\begin{itemize}
			\item The application shall ensure that all app data be secured and encrypted
			\item The application should use OAuth for access delegation
		\end{itemize}
	\end{itemize}

	\begin{itemize}
		\item OS Requirements
		\begin{itemize}
			\item The application shall run on both iOS and android devices
			\item The application shall run on all devices newer than android 5.0 (API 21) - around 94.1\% of android devices
			\item The application shall run on all devices newer than iOS 9.0 - around 99.6\% of iOS devices
		\end{itemize}
	\end{itemize}
	
	\subsection{Use Cases}
	Here the use cases are presented, which are described by Ivar Jacobson as “a description of a set of sequences of actions and variants that a system performs that yield an observable result of value to an actor.” (Jacobson, et. al., 1999, p.41). To produce the use cases we reference both the user personas created in \autoref{user-personas} and the functional and non-functional requirements discussed in the previous section.
	
	The following contains the most pertinent use cases, which can be found within \autoref{chap:use-cases}
	


	
	\section{Software Specification}
	In this section, Iwe discuss the structure of the project, the system and software architectures and data and state management
	
	\subsection{System Architecture}
	https://www.raywenderlich.com/6373413-state-management-with-provider
	System Architecture can be broadly defined as a conceptual model which outlines the structure, behaviour and interactions between internal and external components of the system. Modelling and creating a structured and well-defined architecture allows for the development of a sustainable, scalable and stable software product which can easily grow relevant to the demands placed on it's features.
	
	The overall system architecture can be seen below in figure \ref{fig:system-architecture}.
	
	\begin{figure}[H]
		\centering
		\includegraphics[scale=0.7]{images/system-architecture.png}
		\caption{High-level System Architecture}
		\label{fig:system-architecture}
	\end{figure}
	
	When designing the architecture, core business logic was kept separate from the UI, database and network. For example, when interacting with the database the UI called upon the utilities package and any API calls where contained within the providers package. The project was also split into 3 layers, according to clean architecture principles (\cite{Martin, 2017})
	
	\begin{itemize}
		\item Presentation layer
		\begin{itemize}
			\item Screens - Contains unique UI elements
			\item Widgets - Elements that are used to create the UI and feed into the screens
		\end{itemize}
	\end{itemize}
	
	\begin{itemize}
		\item Domain layer
		\begin{itemize}
			\item Utilities - Contains business logic and elements that are used to make calls to the database or interact with any external APIs.
			\item Providers - State management tools that are called on throughout the application.
		\end{itemize}
	\end{itemize}
	
	\begin{itemize}
		\item Data layer
		\begin{itemize}
			\item Models - Contains the models used for local storage.
		\end{itemize}
	\end{itemize}

	 This generally follows clean architecture principles (\cite{Martin, 2017}) and each layer is represented in figure \ref{fig:clean-architecture} below.
	 
	 \begin{figure}[H]
	 	\centering
	 	\includegraphics[scale=0.7]{images/clean-architecture.png}
	 	\caption{Clean Architecture (adapted from (\cite{Martin, 2017}))}
	 	\label{fig:clean-architecture}
	 \end{figure}
	 
	

	
	\subsection{Client-side Specification}

	\subsubsection{Domain Model}
	TODO: discuss domain model
	
	\begin{figure}[H]
		\centering
		\includegraphics[scale=0.7]{images/model-class-diagrams.png}
		\caption{Model Class Diagrams}
		\label{fig:model-class-diagrams}
	\end{figure}
	
	\subsubsection{Deciding on a Framework}
	There exists a variety of software architectural frameworks, although not exhaustive this includes Client-server, Peer-to-peer, Microservices and Model-view controller. For brevity
	
	(INSERT SYSTEM ARCHITECTURE DIAGRAM)
	https://www.researchgate.net/profile/Wei-Jung-Shiang/publication/267418345/figure/fig1/AS:295656959823872@1447501520198/Architecture-for-XML-based-data-exchange-model.png - 
	
	https://www.researchgate.net/figure/Overview-of-the-schema-matching-system\_fig2\_267418345
	Good example
	
	\subsubsection{State Management}	
	Simple state management used (Provider) -
	involves: ChangeNotifier and Provider.of
	https://flutter.dev/docs/development/data-and-backend/state-mgmt/simple
	
	PARENTBARBERSPROVIDER provides to a lot of classes

	
	\subsection{Server-side Specification}
	
	
	\section{Implementation}
		\label{chap:implementation}
	Here we discuss the process of wire-framing the application, along with the sprints, each of which pertains to relevant feature within the application. Finally, we address each use case individually.
	
	\subsection{Prototyping}
	\subsubsection{Initial Sketches and Brainstorming}
	(UPDATE PHOTO WITH INITIAL SKETCHES)
	
	\subsubsection{Wire-framing}
	An important component of UCD and more generally UX is wire-framing, which involves making a mock-up of the application that acts as an early prototype to influence later development, along with allowing for early beta testing \cite{Arnowitz Arent Berger}. For the wire-framing application Adobe XD was chosen for several reasons. Firstly, it has strong prototyping functionality, allowing the user to click around the application through the use of ‘components’. This interactivity means that early testers can get a real feel for how the application works. An illustration of this can be seen below, whereby each arrow represents a state change in the form of a trigger/ action pair, whereby for example a user could click on ‘Available Right Now’ and be taken to the ‘Checkout’ as seen in figure \ref{fig:prot-comp} below.
	\begin{figure}[H]
		\centering
		\includegraphics[scale=0.5]{images/prototyping-components.png}
		\caption{Component Interactivity within Adobe XD}
		\label{fig:prot-comp}
	\end{figure}
	
	
	Adobe XD also allows for easy distribution of the prototype in the form of a shareable link that opens in the browser and encompasses the same functionality and components that can be found within the application itself, meaning that anyone with access to a browser can test the prototype. Along with this, the prototype also allows for comments to be made, which are fed back to the owner. This comment capability was used early on during beta testing when it was sent out with the early questionnaire and influenced initial design decisions \cite{smashing-magazine}.
	
	When designing the screens there was a strong focus on user experience following Nielsons 10 Heuristics for User Interface Design \cite{nielson-normal-group-2020} . For example, the functionality was kept as minimal as possible to avoid clittering and avoid cognitive load on the user, the user was given control to go back and forward between previous screens to allow for user control and freedom and simple and self-explanatory language was used to apply recognition over recall. For example, the Sign In screen below extraneous text was kept to a minimum by using images for the login items, such as Google, Facebook and Twitter, a sign up button was included to allow the user to access the application through creating a new account and large, clear sign in forms and buttons were used. The full interactive Adobe XD wireframe can be found \href{https://xd.adobe.com/view/c6aeda9c-9b9a-456a-b699-cc4cd8b4cefa-93fd/}{here}. 
	
	\begin{figure}[H]
		\centering
		\includegraphics[scale=0.85]{images/sign-in.png}
		\caption{Sign In Page Made With Adobe XD}
		\label{fig:sign-in}
	\end{figure}
	
	
	\subsection{Sprint 1 - Setup}
	During the first sprint, the task involved setting up the environment in preparation to begin development. For the editor, Android Studio was 
	TALK ABOUT SCREENS/ MODELS/ HELPER etc
	\subsubsection{File Structure}
	Although there is no official recommendation for structuring the app, here we follow a commonly used structure which includes models; the files that serve as collections of data that are used in conjunction with the widgets to form the user interface of the application; providers; screens; utilities; and widgets; .
	
	
	
	\subsection{Sprint X - UI}
	Flutter offer an Adobe XD plugin to turn wireframes directly into code, however, this was not used for several reasons. In Adobe XD components are positioned absolutely, whereas in Flutter it is done relatively, leading to several issues with positioning. Adobe XD also does not contain customer properties and therefore mapping these to components, such as title is not possible.
	
	(DISCUSS HOW FOLLOWED WIREFRAME AND COMPARE IMAGES TO REAL APPLICATION)
	
	\subsubsection{Cupertino vs Material}
	The final project was built using material due to..
	
	\subsection{Sprint X - Login and Sign Up}
	
	\subsubsection{Authentication}
	\label{authentication}
	Firebase has its own built in authentication (Firebase Authentication) which was used for the project due to several considerations -
	\begin{itemize}
		\item Excellent built in security features
		\item Integration with firebase database and storage
		\item Easy modification through Googles declarative language
	\end{itemize}
	Within the UserDatabase class, for the createNewUser function, we pass through the authentication uid, which is then used as the document id, so that future calls can refer to this and therefore fetch the document, without doing a call such as 
	\begin{lstlisting}
		_firebaseFirestore.collection(collection).where('uid', .isEqualTo('givenID')).get()
	\end{lstlisting}
	which does not scale well due to a search time of $\mathcal{O}$(n). Instead, we store the uid as the document id, allowing us to do a similar, although quicker call such as
	\begin{lstlisting}
		_firebaseFirestore.collection(collection).doc(userId).get()
	\end{lstlisting}
	which gives a search time of $\mathcal{O}$(1).
	
	When authenticating through Google, Facebook and Twitter, a user is created so that data can be persistently stored alongside the credentials.
	
	
	\subsection{Sprint X - Backend Design}
	
	\subsubsection{Database Design}
	As previously discussed, for the project it was decided to use a noSQL database, instead of a relational one. Despite this, it was decided that a database schema should be constructed to constitute the parent barbers, barbers and products as modelling this way added to readability with the added benefit of the features previously discussed. The schema is represented in figure \ref{fig:database-schema} below.
	( TODO: add orders to the database schema)
	
	\begin{figure}[H]
		\centering
		\includegraphics[scale=0.7]{images/database-schema.png}
		\caption{Database Schema}
		\label{fig:database-schema}
	\end{figure}
	
	Loading all parentBarbers/ barbers and products - 
	When entering the app it is more effecient to make one call to the server rather than multiple due to GPS restrictions. Therefore ParentBarbers, barbers and all of their products are loaded within a certain radius, rather than loading parentBarbers and barbers separately throughout the app. 
	
	Don't nest as calls will fetch all of the parent and nested structured so better to have separate data.
	
	(INSERT DATABASE SCHEMA)
	
	\subsection{Sprint X - Connecting the Frontend and Backend}
	To get the list of items within a range
	1) Get the users current location
	2) Query firestore with the radius
	
	\subsubsection{Loading the barbers}
	When loading the barbers either they were loading through making a database query based on the parentBarberId once the parent barbers were loaded, or every barber was loaded into memory and this was the filtered in memory.
	\\
	Pros of first method - Scales well as does not matter how many barbers there are
	\\
	Cons of first method - more costly as there are more requests to the db
	\\
	Pros of 2nd - cheaper
	\\
	cons - does not scale well
	
	\subsubsection{Shopping Cart}
	Chose to store the cart items in a nested array as this adds to readbility and structure and although this restricts look up time, this is not relevant due to the limited number of items a user will order.
	
	\subsubsection{Checkout}
	Similar to section \ref{authentication} to create a search time of $\mathcal{O}$(1) a seperate 'orders' collection was created, with each document representing all of the pertinent order details. Within each barber and user, the document (order) id was then added to each respective order arrays, for quick and easy lookup.
	
	\subsection{Sprint X - Location}
	https://firebase.google.com/docs/firestore/solutions/geoqueries
	Once the user logs in they give their location in the form of their address. This is stored in 'geohashes', which are longitude and latitiude co-ordinates that are hashed into a single Base32 string. Each character presents a greater level of precision and therefore we opted for 9, which represents an area of 5 x 5 meters.
	
	User location access is granted through the following line in the ./android/app/src/main/AndroidManifest.xml file
	\begin{lstlisting}
		<uses-permission android:name="android.permission.ACCESS_FINE_LOCATION"/>
	\end{lstlisting}
	For the search results we use a drop down menu in the form of flutters built in 'showSearch' function loosely following a guide on medium \cite{medium-comerge} to display a search page and 'SearchDelegate' to define the content of said search page.
	
	A textEditingController is used to collect the inputted data from the user and pass through to the showSearch function.
	
	\noindent
	
	\begin{minted}[obeytabs=true,tabsize=2, linenos]{dart}
		Future<bool> barberSignIn({String email, String password}) async {
			try {
				_authStatus = AuthStatus.AUTHENTICATING;
				notifyListeners();
				await _firebaseAuth.signInWithEmailAndPassword(email: email, 
				password: password);
				barberModel = await _orderUtility.getBarberById(_firebaseAuth
				.currentUser.uid);
				_authStatus = AuthStatus.BARBER_AUTHENTICATED;
				print("signed in barber " + email);
				notifyListeners();
				return true;
			} on FirebaseAuthException catch (e) {
				print('Failed to sign in barber: ' + e.toString());
				_authStatus = AuthStatus.NOT_AUTHENTICATED;
				notifyListeners();
				return false;
			}
		} 
	\end{minted}
	TODO: Remove this example code
	
	
	\subsubsection{Autocomplete locations}
	As a means for the user to autocomplete their address when signing up, the 'Place Autocomplete service' within the Google Places API, which returns location predictions in response to HTTP requests was implemented using a request adhering to a set of parameters, the full list, along with details of the API can be found on the Google Developers website \cite{requests}.
	First, we enable the Places API within the Google console, before we then create a location model which can hold the data returned from the API. We then create an API request using the above aforementioned API format. For brevity, not every option is discussed, but those of importance include 'input', which is the user query, 'types', which determines the query returned, for which we specify address as we wish to fetch the users full address and a session token, which is required for each new query. The query can be seen here:
	\begin{lstlisting}
		'https://maps.googleapis.com/maps/api/place/autocomplete/json?input=$input&types=address&components=country:uk&lang=en&key=$apiKey&sessiontoken=$sessionToken'
	\end{lstlisting}
	The returned results are in json format and after some minor error checking we parse using json.decode into a list with our LocationModel class, whilst assigning a new UUID for each query (Google recommends to use version 4 UUID and so this is used here). Without our 'user\_gps.dart' file
	
	For the content of the search page we use pass in newly created session token into the ShowSearchPage class, which in turn sends an API request and parses the json data to return a list of locations in the form of 'place id's' and 'description' using a FutureBuilder. From here, we pass through the location id to the getLocationDetails function to fetch the address details of each location and put into a PlaceModel object.
	
	Next, we parse the data into JSON format by passing the PlaceModel object into the function 'createLocationMap', which creates a map using the location data. Finally we pass through this map to the 'addLocationDetails', which uses the given user id to update the database with the users location.
	
	\subsection{Sprint X - Barber Side Application}
	In order to create a comprehensive application the final sprint involved coding a barber side app, giving the ability for the barber interact with the database and allow them to create an account, add and remove barbers and view orders.
	
	\subsubsection{Creating a Parent Barber}
	As the location functionality of the application requires both longitude and latitude co-ordinates, along with a geohash (EXPLAIN HOW PLUGIN USED TO GET THIS).
	
	\subsection{Widgets, Common Items and Added Features}
	Here we discuss any items not covered within a specified sprint.
	
	\subsubsection{Widgets}
	Several widgets were used to increase readability and brevity of code. For example, 'return\_text.dart' allows for access to the main components of the Text function and 'return\_image.dart' allows for easy use of the NetworkImage function, giving brevity and readability to the code.
	
	\subsubsection{Common Items}
	The common items contains global variables that were accessible throughout the project. Initially this included structures and arrays that served as objects to test the functionality of the frontend, for example a barber shop class with a nested list of barbers classes, each with a name, age, description etc. As backend functionality was added these items were removed. A theme class was then added which contains dart files that can be implemented. Doing it in this way meant that the application could be easily styled, without any unnecessary refactoring of code.
	
	\subsubsection{Launcher Icons}
	Created using GIMP. Plugin flutter\_launcher\_icons used to install icons across android and iOS
	
	\subsubsection{Discount Codes}
	
	\subsubsection{Hide and Show Password}
	
	\subsection{Use Cases - Implementation}
	
	\section{Testing}
	\subsection{User Testing}
	\subsection{Acceptance Testing}
	\cite{Humble and Farley, 2010, chapter 8}
	

	
	\bibliography{bib file}
	
	\appendix
	\appendixpage
	\section{Use Cases}
	\label{chap:use-cases}
	
	\begin{subappendices}
		
	\subsubsection{Use case 1 - Sign Up}
	
	\begin{table}[H]
		\begin{tabular}{ll}
			\hline
			\rowcolor[HTML]{EFEFEF} 
			\textbf{Use Case X}  & \textbf{Login}                                                 \\ \hline
			\rowcolor[HTML]{F5FBFF} 
			Description 		 & \textit{This is a description}                                 \\ \hline
			\rowcolor[HTML]{EFEFEF} 
			Pre-conditions       & Authentication enumeration status must be set to UNINITIALISED \\ \hline
			\rowcolor[HTML]{F5FBFF} 
			Basic Flow           & This is a basic flow                                           \\ \hline
			\rowcolor[HTML]{EFEFEF} 
			Alternative Paths    & 1) This is a basic flow                                        \\
			\rowcolor[HTML]{EFEFEF} 
			& 2) This is a basic flow                                        \\
			\rowcolor[HTML]{EFEFEF} 
			& 3) This is a basic flow                                        \\
			\rowcolor[HTML]{EFEFEF} 
			& 4) This is a basic flow                                       
		\end{tabular}
	\end{table}

	\subsubsection{Use case 2 - Login}
	
	\begin{table}[H]
		\begin{tabular}{ll}
			\hline
			\rowcolor[HTML]{EFEFEF} 
			\textbf{Use Case X}  & \textbf{Login}                                                 \\ \hline
			\rowcolor[HTML]{F5FBFF} 
			Description 		 & \textit{This is a description}                                 \\ \hline
			\rowcolor[HTML]{EFEFEF} 
			Pre-conditions       & Authentication enumeration status must be set to UNINITIALISED \\ \hline
			\rowcolor[HTML]{F5FBFF} 
			Basic Flow           & This is a basic flow                                           \\ \hline
			\rowcolor[HTML]{EFEFEF} 
			Alternative Paths    & 1) This is a basic flow                                        \\
			\rowcolor[HTML]{EFEFEF} 
			& 2) This is a basic flow                                        \\
			\rowcolor[HTML]{EFEFEF} 
			& 3) This is a basic flow                                        \\
			\rowcolor[HTML]{EFEFEF} 
			& 4) This is a basic flow                                       
		\end{tabular}
	\end{table}

	\subsubsection{Use case 3 - }
	
	\begin{table}[H]
		\begin{tabular}{ll}
			\hline
			\rowcolor[HTML]{EFEFEF} 
			\textbf{Use Case X}  & \textbf{Login}                                                 \\ \hline
			\rowcolor[HTML]{F5FBFF} 
			Description 		 & \textit{This is a description}                                 \\ \hline
			\rowcolor[HTML]{EFEFEF} 
			Pre-conditions       & Authentication enumeration status must be set to UNINITIALISED \\ \hline
			\rowcolor[HTML]{F5FBFF} 
			Basic Flow           & This is a basic flow                                           \\ \hline
			\rowcolor[HTML]{EFEFEF} 
			Alternative Paths    & 1) This is a basic flow                                        \\
			\rowcolor[HTML]{EFEFEF} 
			& 2) This is a basic flow                                        \\
			\rowcolor[HTML]{EFEFEF} 
			& 3) This is a basic flow                                        \\
			\rowcolor[HTML]{EFEFEF} 
			& 4) This is a basic flow                                       
		\end{tabular}
	\end{table}

	\subsubsection{Use case 4 - }
	
	\begin{table}[H]
		\begin{tabular}{ll}
			\hline
			\rowcolor[HTML]{EFEFEF} 
			\textbf{Use Case X}  & \textbf{Login}                                                 \\ \hline
			\rowcolor[HTML]{F5FBFF} 
			Description 		 & \textit{This is a description}                                 \\ \hline
			\rowcolor[HTML]{EFEFEF} 
			Pre-conditions       & Authentication enumeration status must be set to UNINITIALISED \\ \hline
			\rowcolor[HTML]{F5FBFF} 
			Basic Flow           & This is a basic flow                                           \\ \hline
			\rowcolor[HTML]{EFEFEF} 
			Alternative Paths    & 1) This is a basic flow                                        \\
			\rowcolor[HTML]{EFEFEF} 
			& 2) This is a basic flow                                        \\
			\rowcolor[HTML]{EFEFEF} 
			& 3) This is a basic flow                                        \\
			\rowcolor[HTML]{EFEFEF} 
			& 4) This is a basic flow                                       
		\end{tabular}
	\end{table}

	\subsubsection{Use case 5 - }
	
	\begin{table}[H]
		\begin{tabular}{ll}
			\hline
			\rowcolor[HTML]{EFEFEF} 
			\textbf{Use Case X}  & \textbf{Login}                                                 \\ \hline
			\rowcolor[HTML]{F5FBFF} 
			Description 		 & \textit{This is a description}                                 \\ \hline
			\rowcolor[HTML]{EFEFEF} 
			Pre-conditions       & Authentication enumeration status must be set to UNINITIALISED \\ \hline
			\rowcolor[HTML]{F5FBFF} 
			Basic Flow           & This is a basic flow                                           \\ \hline
			\rowcolor[HTML]{EFEFEF} 
			Alternative Paths    & 1) This is a basic flow                                        \\
			\rowcolor[HTML]{EFEFEF} 
			& 2) This is a basic flow                                        \\
			\rowcolor[HTML]{EFEFEF} 
			& 3) This is a basic flow                                        \\
			\rowcolor[HTML]{EFEFEF} 
			& 4) This is a basic flow                                       
		\end{tabular}
	\end{table}
	
	\end{subappendices}
	\section{Figures}
	
	\section{Application Images}
	\section{Code Snippets}
	
\end{document}