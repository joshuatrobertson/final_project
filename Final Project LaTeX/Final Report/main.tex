
%----------------------------------------------------------------------------------------
%	PACKAGES AND OTHER DOCUMENT CONFIGURATIONS
%----------------------------------------------------------------------------------------

\documentclass[12pt]{article}
\usepackage[english]{babel}
\usepackage{amsmath}
\usepackage{graphicx}
\usepackage{float}
\usepackage[colorlinks = true,
linkcolor = blue,
urlcolor  = blue,
citecolor = blue,
anchorcolor = blue]{hyperref}
\usepackage{blindtext}


\begin{document}
	
	\begin{titlepage}
		
		\newcommand{\HRule}{\rule{\linewidth}{0.5mm}} 
		
		\center % Center everything on the page
		
		%----------------------------------------------------------------------------------------
		%	HEADING SECTIONS
		%----------------------------------------------------------------------------------------
		\textsc{\Large MSc Project: Mobile Hairdresser Application}\\[5 cm] 

		
		
		\includegraphics[scale=0.05]{images/logo.png}\\[1 cm]
				
		

		
		
		\textsc{\LARGE Joshua Robertson} \\[6 cm]
		
		

		
		
		%----------------------------------------------------------------------------------------
		%	TITLE SECTION
		%----------------------------------------------------------------------------------------
		
		
		
		%----------------------------------------------------------------------------------------
		%	AUTHOR SECTION
		%----------------------------------------------------------------------------------------
		
		
		\begin{center}
			
			\textsc\emph{{“A dissertation submitted to the University of Bristol in accordance with the requirements of the degree of Master of Science by advanced study in Computer Science in the Faculty of Engineering."}} \\[1.2 cm]
			
			School of Computer Science, Electrical and Electronic Engineering, and Engineering Maths (SCEEM) \\[1 cm]
			
			
			
		\end{center}
		
		
		
		
		\vfill % Fill the rest of the page with whitespace
		
	\end{titlepage}
	
	\section{Introduction}
	The COVID pandemic has brought with it a shift in perceptions around leaving the home and with that a desire for more homeworking and access to remote services. For example, remote workers show an increase in job satisfaction \cite{flexjobs, 2019}, \cite{CNBC, 2020}, are more productive, have better mental health \cite(flexjobs, 2020) and even make more money (ADD CITE).
	
	New Product Development refers to the entirety of processes leading to bringing a product to market and encompasses several steps as seen in figure \ref{fig:npd} below.
	\newline
	
	\begin{figure}[H]
		\centering
		\includegraphics[scale=0.15]{images/npd.png}
		\caption{The 7 Steps of New Product Development}
		\label{fig:npd} \cite{shopify}
	\end{figure}
	
	
	\subsection{Ideation and Concept}
	\blindtext
	
	\subsection{Market Analysis}
	In order to gauge whether there is a market for the proposed analysis, a survey was carried out in which users were asked about whether they could see themselves using the application features, among other things.
	 
	\subsubsection{Existing Applications}
	
	
	\subsubsection{The Target User}
	
	\subsubsection{Programming Language}
	When deciding on the programming software, several metrics were taken into consideration, including cross-platform functionality, speed, speed of development and performance. For this reason, Dart and the corresponding Flutter software development kit (SDK) were chosen for the primary software. Flutter is a cross-platform development kit, meaning that it will natively run on both iOS and android applications created by Google \cite{flutter}. Dart is compiled ahead-of-time into native ARM code giving better performance compared to other similar development kits, such as React Native and the user interface  is implemented within a fast, low-level C++ library giving great speed to the application. Dart has also seen a large increase in usage within recent years, jumping up 532\% from 2018 to 2019 \cite{Github, 2018} meaning that there is now an extensible list of third-party plugins available and a large community.
	

	\subsection{User Personas}
	The creation of user personas representing fictitious, archetypal users is an essential part of application development \cite{Grudin and Pruitt, 2002} and allows a deep understanding of the target user to be sought and implemented within the features and design of the application \cite{Long, 2009}. Although there are some shortcomings to qualitative persona generation, such as validity concerns and user bias \cite{Chapman and Milham, 2007} which are addressed by other methods, such as data-driven personas \cite{Mcginn and Kotamraju, 2008}, we have decided to stick with qualitative methods, which allow for enough brevity and depth for the scope of the project. Here we created 3 personas, which are discussed in detail below.
	\begin{itemize}
		\item Persona 1: 
		
		INSERT PERSONA INFO
		\item Persona 2:
		\item Persona 3:
	\end{itemize}


	\section{Production}
	\subsection{Version Control}
	
	\subsection{Sprint 1 - Setup}
	During the first sprint, the task involved setting up the environment in preparation to begin development. For the editor, Android Studio was 
	
	\subsection{Sprint X - UI}
	Flutter offer an Adobe XD plugin to turn wireframes directly into code, however, this was not used for several reasons. In Adobe XD components are positioned absolutely, whereas in Flutter it is done relatively, leading to several issues with positioning. Adobe XD also does not contain customer properties and therefore mapping these to components, such as title is not possile.
	

	
\end{document}